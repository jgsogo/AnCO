% This is LLNCS.DEM the demonstration file of
% the LaTeX macro package from Springer-Verlag
% for Lecture Notes in Computer Science,
% version 2.4 for LaTeX2e as of 16. April 2010
%
\documentclass{llncs}
%
\usepackage{makeidx}  % allows for indexgeneration
%
\begin{document}
%
\mainmatter              % start of the contributions
%
\title{Búsqueda de Camino Más Corto en Grafos Vastos Dinámicos Utilizando Colonias de Hormigas (Trabajo en Curso)}
%
\titlerunning{Collaborative ACO}  % abbreviated title (for running head)
%                                     also used for the TOC unless
%                                     \toctitle is used
%
\author{Javier Garc{\'i}a Sogo}
%
\authorrunning{Javier Garc{\'i}a Sogo} % abbreviated author list (for running head)
%
%%%% list of authors for the TOC (use if author list has to be modified)
\tocauthor{Javier Garc{\'i}a Sogo}
%
\institute{Universidad Polit{\'e}cnica de Madrid, Madrid, España}

\maketitle              % typeset the title of the contribution

\begin{abstract}
Ant Colony Optimization is a well known metaheuristic applied to solve a large number of problems presented in a graph. 
. \dots
\keywords{Ant Colony Optimization, huge graph, dynamic graph}
\end{abstract}
%
\section{Introduction}
%
Los grafos se han utilizado para plantear distintos tipos de problemas de muy variada complejidad desde que aparecieron por vez primera.
Muchos de estos problemas se enfocan de tal forma que su solución consiste en encontrar caminos entre dos nodos cualesquiera de un grafo. 
Ante esta situación han aparecido muchos artículos aportando algoritmos y metodologías al estado del arte, entre ellos se encuentran los algoritmos basados en colonias de hormigas \cite{Dorigo1992}.

En la actualidad algunos de estos problemas plantean nuevos retos conforme van apareciendo nuevos requisitos: los grafos cada vez son mayores, la estructura de los mismos puede ser cambiante con nuevos nodos y enlaces que aparecen o son eliminados, e incluso el coste asociado a cada arista puede variar también.
Además, a todo esto hay que unir el tiempo de respuesta, un requisito que cada vez es más importante, y que en muchas aplicaciones en tiempo real o limitado se convierte en la principal restricción, relegando a un segundo plano la obtención del camino óptimo \cite{Rivero2011}.





%
\subsection{Collaborative ACO}
%

\section{Conclusions and future work}

% FUTURE WORK
% Casos en los que hay que atravesar una colonia en grafos no bidireccionales

% Tamaño del grafo --> base de datos

% Movimiento de las colonias - prominencia


%
% ---- Bibliography ----
%

\begin{thebibliography}{5}
%
\bibitem {Dorigo1992}
Dorigo, M.:
Optimization, learning and natural algorithms.
Doctoral Thesis, Dipartamento di Elettronica, Politecnico di Milano, Italy. (1992)

\bibitem {clar:eke}
Clarke, F., Ekeland, I.:
Nonlinear oscillations and
boundary-value problems for Hamiltonian systems.
Arch. Rat. Mech. Anal. 78, 315--333 (1982)

\bibitem {clar:eke:2}
Clarke, F., Ekeland, I.:
Solutions p\'{e}riodiques, du
p\'{e}riode donn\'{e}e, des \'{e}quations hamiltoniennes.
Note CRAS Paris 287, 1013--1015 (1978)

\bibitem {mich:tar}
Michalek, R., Tarantello, G.:
Subharmonic solutions with prescribed minimal
period for nonautonomous Hamiltonian systems.
J. Diff. Eq. 72, 28--55 (1988)

\bibitem {tar}
Tarantello, G.:
Subharmonic solutions for Hamiltonian
systems via a $\bbbz_{p}$ pseudoindex theory.
Annali di Matematica Pura (to appear)

\bibitem {rab}
Rabinowitz, P.:
On subharmonic solutions of a Hamiltonian system.
Comm. Pure Appl. Math. 33, 609--633 (1980)

\end{thebibliography}



\end{document}
