% This is LLNCS.DEM the demonstration file of
% the LaTeX macro package from Springer-Verlag
% for Lecture Notes in Computer Science,
% version 2.4 for LaTeX2e as of 16. April 2010
%
\documentclass{llncs}
%
\usepackage{makeidx}  % allows for indexgeneration
%
\begin{document}
%
\mainmatter              % start of the contributions
%
\title{B{\'u}squeda de Camino M{\'a}s Corto en Grafos Vastos Din{\'a}micos Utilizando Colonias de Hormigas (Trabajo en Curso)}
%
\titlerunning{Collaborative ACO}  % abbreviated title (for running head)
%                                     also used for the TOC unless
%                                     \toctitle is used
%
\author{Javier Garc{\'i}a Sogo}
%
\authorrunning{Javier Garc{\'i}a Sogo} % abbreviated author list (for running head)
%
%%%% list of authors for the TOC (use if author list has to be modified)
\tocauthor{Javier Garc{\'i}a Sogo}
%
\institute{Universidad Polit{\'e}cnica de Madrid, Madrid, Espa\~{n}a}

\maketitle              % typeset the title of the contribution

\begin{abstract}
Los algoritmos basados en colonias de hormigas han sido utilizados con éxito para resolver el problema de b{\'u}squeda de caminos en grafos est{\'a}ticos y din{\'a}micos. Sin embargo su aplicaci{\'o} a grafos grandes se ve limitada por el tiempo que tardan en converger.
En este documento se presenta un nuevo\footnote{La "novedad" queda supeditada a un an{\'a}lisis exhaustivo del estado del arte.} algoritmo que se apoya en un conjunto de colonias de hormigas para construir un meta-grafo que permite dividir el problema.
\keywords{Ant Colony Optimization, huge graph, dynamic graph}
\end{abstract}
%
\section{Introduction}
%
Los grafos se han utilizado para plantear distintos tipos de problemas de muy variada complejidad desde que aparecieron por vez primera.
Muchos de estos problemas se enfocan de tal forma que su soluci{\'o}n consiste en encontrar caminos entre dos nodos cualesquiera de un grafo. 
Ante esta situaci{\'o}n han aparecido muchos art{\'i}culos aportando algoritmos y metodolog{\'i}as al estado del arte, entre ellos se encuentran los algoritmos basados en colonias de hormigas \cite{Dorigo1992}.

En la actualidad algunos de estos problemas plantean nuevos retos conforme van apareciendo nuevos requisitos: grafos cada vez son mayores, estructura din{\'a}mica con nuevos nodos y enlaces que aparecen o son eliminados, e incluso el coste asociado a cada arista puede variar tambi{\'e}n.
Adem{\'a}s, a todo esto hay que unir el tiempo de respuesta, un requisito que cada vez es m{\'a}s importante, y que en muchas aplicaciones en tiempo real o limitado se convierte en la principal restricci{\'o}n, relegando a un segundo plano la obtenci{\'o}n del camino {\'o}ptimo \cite{Rivero2011}.

Los algoritmos de colonias de hormigas han demostrado su capacidad de adaptaci{\'o}n a entornos din{\'a}micos y su eficiencia a la hora de encontrar caminos entre nodos.
Sin embargo, su aplicaci{\'o}n a grafos de gran tama\~{n}o es limitada y en la literatura la mayor{\'i}a de propuestas que se encuentran se basan en realizar un pre-procesado del grafo para limitar las b{\'u}squedas en vez de hacerlo en el grafo completo, perdiendo en muchas ocasiones la adaptabilidad a los cambios.

Nuestra propuesta surge tras la lectura de la tesis doctoral de Rivero \cite{Rivero2011} que propone el algoritmo \textit{SoSACO} (\textit{Sense of Smell ACO}) en el que introduce un conjunto de \textit{Nodos Comida} y \textit{Olor a Comida} para ayudar a las hormigas en la b{\'u}squeda del camino.
En este trabajo planteamos un enfoque diferente, aunque los objetivos que perseguimos son los mismos:
\begin{enumerate}
  \item Tiempo de respuesta m{\'i}nimo: en caso de que exista una camino entre los nodos planteados, el algoritmo debe dar una respuesta afirmativa r{\'a}pidamente, que luego podr{\'a} intentar mejorar.
  \item Grafo de aplicaci{\'o}n \textit{vasto} (de alta cardinalidad): el algoritmo debe poder adaptarse a un tama\~{n}o de grafo cualquiera.
  \item Adaptaci{\'o}n a grafos din{\'a}micos: el dinamismo del grafo no debe suponer un obst{\'a}culo para la ejecuci{\'o}n del algoritmo.
  \item Topolog{\'i}a gen{\'e}rica: el algoritmo debe dise\~{n}arse para poder trabajar sobre cualquier topolog{\'i}a de grafo.
\end{enumerate}

Estos objetivos son muy ambiciosos para un trabajo de estas caracter{\'i}sticas, se ha hecho un esfuerzo importante en desarrollar el concepto para poder probarlo, pero apenas se han podido realizar pruebas ni an{\'a}lisis de los par{\'a}metros.
Sin embargo creemos que el planteamiento puede ser prometedor (pendiente de una revisi{\'o}n exhaustiva del estado del arte) para satisfacer el problema planteado.
Sin duda alguna, {\'e}ste debe ser considerado un trabajo en curso que ser{\'a} ampliado en el futuro.







% ALGORITMO
\section{Algoritmo AnCO}
La idea principal del algoritmo \textit{AnCO} (\textit{Ant N-Colonies Optimization}) consiste en distribuir $n$ colonias de hormigas sobre el grafo original y construir un \textit{meta-grafo} donde las colonias son los nodos y los arcos se construyen con sus relaciones de vecindad.
Este meta-grafo permitir{\'a} dividir la b{\'u}squeda del camino entre dos nodos en varios problemas m{\'a}s peque\~{n}os sin p{\'e}rdida de generalidad.

En los siguientes apartados describiremos con detalle el funcionamiento del algoritmo en el proceso de b{\'u}squeda entre un nodo inicial, $n_s$, y otro nodo, $n_e$.

\subsection{Construcci{\'o}n del \textit{meta-grafo}}
Idealmente el algoritmo estar{\'i}a corriendo desde antes de la primera b{\'u}squeda, realizando un proceso de \textit{aprendizaje} que luego utilizaremos.

Se generan $n$ colonias en los nodos $n_1, n_2,..., n_n$ elegidos aleatoriamente y se deja que sus hormigas se muevan $s$ pasos por el grafo libremente (sin ning{\'u}n condicionamiento a la hora de elegir los arcos, tan s{\'o}lo evitando los ya visitados).
Al final de cada iteraci{\'o}n se deposita la feromona correspondiente a cada colonia, $f_i$, en las aristas visitadas por cada de las hormigas y se evapora la cantidad correspondiente seg{\'u}n el ACO cl{\'a}sico.

Al final de cada iteraci{\'o}n cada colonia calcula su \textit{vecindario}: los nodos que han alcanzado sus hormigas y la distancia en pasos m{\'i}nima hasta ellos; esta informaci{\'o}n se utiliza para construir el meta-grafo y actualizar el coste asociado a sus arcos.
Se crear{\'a} un arco $n_i \rightarrow n_k$ cuando las hormigas de $n_i$ hayan encontrado en su camino feromona $f_k$ y a este arco se le asignar{\'a} un coste que denominamos \textit{proximidad} y que cambiar{\'a} a lo largo del tiempo pudiendo llegar a eliminarse el arco si la colonia $n_i$ no vuelve a ver rastro de $n_k$.

\subsubsection{Medidas de proximidad}
Esta m{\'e}trica constituye uno de los puntos clave del algoritmo AnCO, se han utilizado dos aproximaciones:
\begin{itemize}
  \item En primer lugar se intent{\'o} desarrolla una medida absoluta para la proximidad entre dos subgrafos tomando como par{\'a}metros las distancias (vecindario) a los nodos donde se ha encontrado feromona $n_k$ y la concentraci{\'o}n de {\'e}sta. El desarrollo de esta medida constituye parte del trabajo futuro.
  \item Posteriormente se simplific{\'o} el c{\'a}lculo utilizando una medida probabil{\'i}stica que expresara la probabilidad de encontrar la colonia $n_k$, calculada como la proporci{\'o}n entre el n{\'u}mero de hormigas de la colonia $n_i$ que la hab{\'i}an encontrado y el n{\'u}mero de hormigas total de dicha colonia.
\end{itemize}
El c{\'o}digo del algoritmo permite intercambiar con facilidad una rutina de proximidad por otra, al igual que permite utilizar diferentes implementaciones de ACO.
La modularibilidad es un principio que se ha intentado respetar en todo momento.

Utilizando estas m{\'e}tricas se puede construir r{\'a}pidamente un meta-grafo din{\'a}mico que permite encontrar caminos entre dos zonas del grafo original y asignarles un valor de distancia/confianza \textit{a priori}.
En el pr{\'o}ximo apartado veremos c{\'o}mo se realiza este proceso.

\subsection{Existencia de camino}
El algoritmo AnCO est{\'a} pensado para estar ejecut{\'a}ndose continuamente sobre el grafo original, las colonias de hormigas actualizar{\'a}n su vecindario continuamente, y los costes y la topolog{\'i}a del meta-grafo se ver{\'a}n afectados.

A este algoritmo en ejecuci{\'o}n se le podr{\'a} pedir el camino m{\'a}s corto entre $n_s$ y $n_e$, la soluci{\'o}n ser{\'a} construida seg{\'u}n los siguientes pasos:

\begin{enumerate}
  \item Se crear{\'a} una colonia en $n_s$ con comportamiento aleatorio que explorar{\'a} el grafo en busca de las colonias vecinas del meta-grafo.
  \item El mismo procedimiento ser{\'a} utilizado a partir de $n_e$.
  \item Una vez que se han encontrado los vecinos de $n_s$ y $n_e$ se utilizar{\'a} el meta-grafo para calcular la secuencia de colonias {\'o}ptima entre el origen y el destino seg{\'u}n las medidas de proximidad. Con este c{\'a}lculo podr{\'i}amos afirmar que el problema tiene soluci{\'o}n (ver conclusiones) en un tiempo breve.
  \item En este momento el hormiguero de $n_s$ se reinicia implementando cualquier variante de ACO convencional para ir construyendo una soluci{\'o}n incrementalmente.
\end{enumerate}

\subsection{Construcci{\'o}n del camino}
Una vez que sabemos que existe un camino entre los nodos propuestos se reinicia el hormiguero de $n_s$ con una implementaci{\'o}n de ACO que busca de forma voraz el rastro de feromona de los objetivos priorizando los m{\'a}s pr{\'o}ximos al nodo final, $n_e$.
Estas hormigas se comportan utilizando la variante MMAS del algoritmo ACO hasta que encuentran feromona correspondiente a uno de los objetivos, en ese momento su comportamiento cambia y empiezan a seguir el gradiente de concentraci{\'o}n de feromona de esa colonia hasta llegar al nodo donde se encuentra la colonia o encontrar feromona del siguiente objetivo.
Cuando alcanzan el nodo de una colonia, o feromona de un objetivo posterior, se elimina todos los objetivos anteriores para que no vuelvan hacia atr{\'a}s en el espacio de b{\'u}squeda.








% CONCLUSIONES
\section{Conclusiones y trabajo futuro}
Como hemos indicado en la introducci{\'o}n, este documento plantea una propuesta de algoritmo que debe ser discutida y desarrollada con mucha m{\'a}s profundidad antes de poder ofrecer conclusiones sobre el mismo.
Sin embargo, las pruebas que se han podido realizar muestran que el algoritmo funciona y la idea puede ser v{\'a}lida, aunque quedan muchas l{\'i}neas abiertas que deben ser investigadas.

En primer lugar debe fijarse una metodolog{\'i}a que permita evaluar el comportamiento de este algoritmo frente a otros presentes en el estado del arte y debe dise\~{n}arse tambi{\'e}n un m{\'e}todo experimental que permita modificar los par{\'a}metros y valorar las distintas soluciones para poder compararlas: diferentes implementaciones de ACO para cada colonia y variaci{\'o}n de sus par{\'a}metros.

Sin embargo, m{\'a}s interesantes resultan otros problemas que surgen del algoritmo que estamos presentando aqu{\'i} y que deben ser abordados:
\begin{enumerate}
  \item Grafos dirigidos: en este experimento se han utilizado grafos no dirigidos que permiten asegurar la conexi{\'o}n entre las colonias del vecindario y tambi{\'e}n entre todos los nodos que pertenecen a una misma colonia.
  \item Posici{\'o}n y n{\'u}mero de colonias del meta-grafo: las conexiones generadas en el meta-grafo condicionan los caminos que van a ser explorados en el grafo subyacente. Una mejora del algoritmo AnCO incluir{\'i}a la creaci{\'o}n de nuevas colonias en zonas despobladas y el desplazamiento de alguna de ellas cuando se encuentren muy pr{\'o}ximas, para ello se considera la creaci{\'o}n de una m{\'e}trica de \textit{prominencia} que tenga en cuenta la visibilidad de los nodos (n{\'u}mero de arcos) y la distancia al resto de colonias dada por la \textit{proximidad}.
  \item Tama\~{n}o del grafo: a pesar de las crecientes capacidades de los ordenadores, el tama\~{n}o del grafo puede no ser abordable por un {\'u}nico ordenador, el algoritmo AnCO se ha desarrollado pensando en que los datos del grafo podr{\'i}an almacenarse en una base de datos \cite{Rivero2011} y as{\'i} el algoritmo no s{\'o}lo ser{\'i}a mucho m{\'a}s ligero, sino que podr{\'i}a ejecutarse de forma distribuida.
  \item Pruebas con grafos din{\'a}micos: las pruebas que se han realizado no incluyen dinamismo en los grafos, en futuros desarrollos deber{\'a} incorporarse esta circunstancia para validar el concepto.
\end{enumerate}

A pesar de que el algoritmo AnCO se encuentra en una fase de concepto, creemos firmemente que su planteamiento es v{\'a}lido para abordar el problema que se planteaba en la introducci{\'o}n; en los pr{\'o}ximos meses se intentar{\'a} mantener activo el desarrollo para conseguir una primera versi{\'o}n que pueda ser competitiva frente a otros planteamientos.



%
% ---- Bibliography ----
%

\begin{thebibliography}{5}
%
\bibitem {Dorigo1992}
Dorigo, M.:
Optimization, learning and natural algorithms.
Doctoral Thesis, Dipartamento di Elettronica, Politecnico di Milano, Italy. (1992)

\bibitem {clar:eke}
Clarke, F., Ekeland, I.:
Nonlinear oscillations and
boundary-value problems for Hamiltonian systems.
Arch. Rat. Mech. Anal. 78, 315--333 (1982)

\bibitem {clar:eke:2}
Clarke, F., Ekeland, I.:
Solutions p\'{e}riodiques, du
p\'{e}riode donn\'{e}e, des \'{e}quations hamiltoniennes.
Note CRAS Paris 287, 1013--1015 (1978)

\bibitem {mich:tar}
Michalek, R., Tarantello, G.:
Subharmonic solutions with prescribed minimal
period for nonautonomous Hamiltonian systems.
J. Diff. Eq. 72, 28--55 (1988)

\bibitem {tar}
Tarantello, G.:
Subharmonic solutions for Hamiltonian
systems via a $\bbbz_{p}$ pseudoindex theory.
Annali di Matematica Pura (to appear)

\bibitem {rab}
Rabinowitz, P.:
On subharmonic solutions of a Hamiltonian system.
Comm. Pure Appl. Math. 33, 609--633 (1980)

\end{thebibliography}



\end{document}
